\documentclass[
    10pt,
    a4paper,
    oneside,
    headinclude,footinclude,
    BCOR5mm,
]{scrartcl}

\input{structure.tex}

\hyphenation{Fortran hy-phe-ation}

\title{\normalfont{\spacedallcaps{Ready-Player-One}}}
\subtitle{Rapport du groupe Harpon}

\author{Martin Lehoux, Pierre Biret \and Sacha Seksik, Loïc Audoin}
\date{\today}

\begin{document}
    
\renewcommand{\sectionmark}[1]{\markright{\spacedlowsmallcaps{#1}}}
\lehead{\mbox{\llap{\small\thepage\kern1em\color{halfgray} \vline}\color{halfgray}\hspace{0.5em}\rightmark\hfil}}
\pagestyle{scrheadings}

\maketitle
% \setcounter{tocdepth}{2}
% \tableofcontents
% \listoffigures
% \listoftables

\section*{Abstract}

\newpage
\section{Réalisation du perceptron}

\paragraph{Semaine jusqu'au 17 octobre}


La semaine a été consacrée à l'amélioration des premier résultats obtenus sur XOR, mais surtout sur MNIST.

% TODO: Ajouter la référence de ce conseil
L'utilisation de la tangente hyperbolique $f(x) = 1.7159 \times tanh(\frac{2x}{3})$ a permis d'améliorer les
résultats, principalement en accélérant les calculs de propagation par rapport à $f(x) = \frac{1}{1 + e^{-x}}$.

Les fonctions permettant de sauvegarder et de charger un perceptron après apprentissage dans un fichier ont été rajoutées.
Mais la structure de donnée de l'autre groupe n'est pas forcément identique, il faut donc se mettre d'accord sur un format de fichier commun.
Il sera ensuite possible de vérifier le comportement de propagation des deux perceptrons, qui devrait être identique.

\renewcommand{\refname}{\spacedlowsmallcaps{References}} % For modifying the bibliography heading
\bibliographystyle{unsrt}
\bibliography{sample.bib} % The file containing the bibliography

\end{document}